\documentclass[]{article}
\usepackage{lmodern}
\usepackage{amssymb,amsmath}
\usepackage{ifxetex,ifluatex}
\usepackage{fixltx2e} % provides \textsubscript
\ifnum 0\ifxetex 1\fi\ifluatex 1\fi=0 % if pdftex
  \usepackage[T1]{fontenc}
  \usepackage[utf8]{inputenc}
\else % if luatex or xelatex
  \ifxetex
    \usepackage{mathspec}
  \else
    \usepackage{fontspec}
  \fi
  \defaultfontfeatures{Ligatures=TeX,Scale=MatchLowercase}
\fi
% use upquote if available, for straight quotes in verbatim environments
\IfFileExists{upquote.sty}{\usepackage{upquote}}{}
% use microtype if available
\IfFileExists{microtype.sty}{%
\usepackage[]{microtype}
\UseMicrotypeSet[protrusion]{basicmath} % disable protrusion for tt fonts
}{}
\PassOptionsToPackage{hyphens}{url} % url is loaded by hyperref
\usepackage[unicode=true]{hyperref}
\hypersetup{
            pdftitle={Activity Monitoring - Reproducible Research},
            pdfauthor={Siddharth Gusain},
            pdfborder={0 0 0},
            breaklinks=true}
\urlstyle{same}  % don't use monospace font for urls
\usepackage[margin=1in]{geometry}
\usepackage{color}
\usepackage{fancyvrb}
\newcommand{\VerbBar}{|}
\newcommand{\VERB}{\Verb[commandchars=\\\{\}]}
\DefineVerbatimEnvironment{Highlighting}{Verbatim}{commandchars=\\\{\}}
% Add ',fontsize=\small' for more characters per line
\usepackage{framed}
\definecolor{shadecolor}{RGB}{248,248,248}
\newenvironment{Shaded}{\begin{snugshade}}{\end{snugshade}}
\newcommand{\KeywordTok}[1]{\textcolor[rgb]{0.13,0.29,0.53}{\textbf{#1}}}
\newcommand{\DataTypeTok}[1]{\textcolor[rgb]{0.13,0.29,0.53}{#1}}
\newcommand{\DecValTok}[1]{\textcolor[rgb]{0.00,0.00,0.81}{#1}}
\newcommand{\BaseNTok}[1]{\textcolor[rgb]{0.00,0.00,0.81}{#1}}
\newcommand{\FloatTok}[1]{\textcolor[rgb]{0.00,0.00,0.81}{#1}}
\newcommand{\ConstantTok}[1]{\textcolor[rgb]{0.00,0.00,0.00}{#1}}
\newcommand{\CharTok}[1]{\textcolor[rgb]{0.31,0.60,0.02}{#1}}
\newcommand{\SpecialCharTok}[1]{\textcolor[rgb]{0.00,0.00,0.00}{#1}}
\newcommand{\StringTok}[1]{\textcolor[rgb]{0.31,0.60,0.02}{#1}}
\newcommand{\VerbatimStringTok}[1]{\textcolor[rgb]{0.31,0.60,0.02}{#1}}
\newcommand{\SpecialStringTok}[1]{\textcolor[rgb]{0.31,0.60,0.02}{#1}}
\newcommand{\ImportTok}[1]{#1}
\newcommand{\CommentTok}[1]{\textcolor[rgb]{0.56,0.35,0.01}{\textit{#1}}}
\newcommand{\DocumentationTok}[1]{\textcolor[rgb]{0.56,0.35,0.01}{\textbf{\textit{#1}}}}
\newcommand{\AnnotationTok}[1]{\textcolor[rgb]{0.56,0.35,0.01}{\textbf{\textit{#1}}}}
\newcommand{\CommentVarTok}[1]{\textcolor[rgb]{0.56,0.35,0.01}{\textbf{\textit{#1}}}}
\newcommand{\OtherTok}[1]{\textcolor[rgb]{0.56,0.35,0.01}{#1}}
\newcommand{\FunctionTok}[1]{\textcolor[rgb]{0.00,0.00,0.00}{#1}}
\newcommand{\VariableTok}[1]{\textcolor[rgb]{0.00,0.00,0.00}{#1}}
\newcommand{\ControlFlowTok}[1]{\textcolor[rgb]{0.13,0.29,0.53}{\textbf{#1}}}
\newcommand{\OperatorTok}[1]{\textcolor[rgb]{0.81,0.36,0.00}{\textbf{#1}}}
\newcommand{\BuiltInTok}[1]{#1}
\newcommand{\ExtensionTok}[1]{#1}
\newcommand{\PreprocessorTok}[1]{\textcolor[rgb]{0.56,0.35,0.01}{\textit{#1}}}
\newcommand{\AttributeTok}[1]{\textcolor[rgb]{0.77,0.63,0.00}{#1}}
\newcommand{\RegionMarkerTok}[1]{#1}
\newcommand{\InformationTok}[1]{\textcolor[rgb]{0.56,0.35,0.01}{\textbf{\textit{#1}}}}
\newcommand{\WarningTok}[1]{\textcolor[rgb]{0.56,0.35,0.01}{\textbf{\textit{#1}}}}
\newcommand{\AlertTok}[1]{\textcolor[rgb]{0.94,0.16,0.16}{#1}}
\newcommand{\ErrorTok}[1]{\textcolor[rgb]{0.64,0.00,0.00}{\textbf{#1}}}
\newcommand{\NormalTok}[1]{#1}
\usepackage{graphicx,grffile}
\makeatletter
\def\maxwidth{\ifdim\Gin@nat@width>\linewidth\linewidth\else\Gin@nat@width\fi}
\def\maxheight{\ifdim\Gin@nat@height>\textheight\textheight\else\Gin@nat@height\fi}
\makeatother
% Scale images if necessary, so that they will not overflow the page
% margins by default, and it is still possible to overwrite the defaults
% using explicit options in \includegraphics[width, height, ...]{}
\setkeys{Gin}{width=\maxwidth,height=\maxheight,keepaspectratio}
\IfFileExists{parskip.sty}{%
\usepackage{parskip}
}{% else
\setlength{\parindent}{0pt}
\setlength{\parskip}{6pt plus 2pt minus 1pt}
}
\setlength{\emergencystretch}{3em}  % prevent overfull lines
\providecommand{\tightlist}{%
  \setlength{\itemsep}{0pt}\setlength{\parskip}{0pt}}
\setcounter{secnumdepth}{0}
% Redefines (sub)paragraphs to behave more like sections
\ifx\paragraph\undefined\else
\let\oldparagraph\paragraph
\renewcommand{\paragraph}[1]{\oldparagraph{#1}\mbox{}}
\fi
\ifx\subparagraph\undefined\else
\let\oldsubparagraph\subparagraph
\renewcommand{\subparagraph}[1]{\oldsubparagraph{#1}\mbox{}}
\fi

% set default figure placement to htbp
\makeatletter
\def\fps@figure{htbp}
\makeatother


\title{Activity Monitoring - Reproducible Research}
\author{Siddharth Gusain}
\date{October 1, 2020}

\begin{document}
\maketitle

\subsection{Loading and preprocessing the
data}\label{loading-and-preprocessing-the-data}

We will start by downloading the data from
\href{https://d396qusza40orc.cloudfront.net/repdata\%2Fdata\%2Factivity.zip}{Activity
monitoring data}, the data was downloaded at 2020-10-02.

Methodology for downloading and preprocessing is given below.

\begin{Shaded}
\begin{Highlighting}[]
\KeywordTok{download.file}\NormalTok{(}\StringTok{"https://d396qusza40orc.cloudfront.net/repdata%2Fdata%2Factivity.zip"}\NormalTok{,}\DataTypeTok{destfile =} \StringTok{"activity.zip"}\NormalTok{)}
\NormalTok{data <-}\StringTok{ }\KeywordTok{read.csv}\NormalTok{(}\KeywordTok{unz}\NormalTok{(}\StringTok{"activity.zip"}\NormalTok{,}\StringTok{"activity.csv"}\NormalTok{),}\DataTypeTok{stringsAsFactors =} \OtherTok{FALSE}\NormalTok{)}
\KeywordTok{library}\NormalTok{(lubridate)}
\KeywordTok{library}\NormalTok{(dplyr)}
\KeywordTok{library}\NormalTok{(ggplot2)}
\NormalTok{data}\OperatorTok{$}\NormalTok{date<-}\KeywordTok{ymd}\NormalTok{(data}\OperatorTok{$}\NormalTok{date)}
\end{Highlighting}
\end{Shaded}

\subsection{Total number of steps}\label{total-number-of-steps}

The summary of sum, mean and median of steps per day is given below.

\begin{Shaded}
\begin{Highlighting}[]
\NormalTok{steps_summary_date <-}\StringTok{ }\NormalTok{data }\OperatorTok\StringTok{ }\KeywordTok{group_by}\NormalTok{(date) }\OperatorTok\StringTok{ }\KeywordTok{summarise}\NormalTok{(}\DataTypeTok{steps =} \KeywordTok{sum}\NormalTok{ (steps))}
\end{Highlighting}
\end{Shaded}

\begin{verbatim}
## `summarise()` ungrouping output (override with `.groups` argument)
\end{verbatim}

\begin{Shaded}
\begin{Highlighting}[]
\NormalTok{sumsteps <-}\StringTok{ }\KeywordTok{sum}\NormalTok{(steps_summary_date}\OperatorTok{$}\NormalTok{steps,}\DataTypeTok{na.rm =}\NormalTok{ T)}
\NormalTok{meansteps <-}\StringTok{ }\KeywordTok{as.integer}\NormalTok{(}\KeywordTok{mean}\NormalTok{(steps_summary_date}\OperatorTok{$}\NormalTok{steps,}\DataTypeTok{na.rm =}\NormalTok{ T))}
\NormalTok{mediansteps <-}\StringTok{ }\KeywordTok{median}\NormalTok{(steps_summary_date}\OperatorTok{$}\NormalTok{steps,}\DataTypeTok{na.rm =}\NormalTok{ T)}
\end{Highlighting}
\end{Shaded}

As calculated in the code chunk above, the sum is \textbf{570608}, the
mean is \textbf{10766} and the median is \textbf{10765}.

Furthermore, a histogram of the total number of steps per day is given
below:

\begin{Shaded}
\begin{Highlighting}[]
\NormalTok{plotdata <-}\StringTok{ }\NormalTok{data }\OperatorTok\StringTok{ }\KeywordTok{group_by}\NormalTok{(date) }\OperatorTok\StringTok{ }\KeywordTok{summarise}\NormalTok{(}\DataTypeTok{steps =} \KeywordTok{sum}\NormalTok{(steps))}
\KeywordTok{hist}\NormalTok{(plotdata}\OperatorTok{$}\NormalTok{steps, }\DataTypeTok{breaks =} \DecValTok{15}\NormalTok{, }\DataTypeTok{xlab =} \StringTok{"Number of steps per day"}\NormalTok{, }\DataTypeTok{main =} \StringTok{"Histogram of daily steps per day"}\NormalTok{, }\DataTypeTok{col =} \StringTok{"green"}\NormalTok{)}
\KeywordTok{rug}\NormalTok{(plotdata}\OperatorTok{$}\NormalTok{steps)}
\end{Highlighting}
\end{Shaded}

\includegraphics{Activity-monitoring-data_files/figure-latex/unnamed-chunk-3-1.pdf}

\subsection{Average daily activity
pattern}\label{average-daily-activity-pattern}

A time series plot of 5-minute intervals against average number of steps
taken across those intervals is given below.

\begin{Shaded}
\begin{Highlighting}[]
\NormalTok{activitypatterndata <-}\StringTok{ }\NormalTok{data }\OperatorTok\StringTok{ }\KeywordTok{select}\NormalTok{(steps,interval) }\OperatorTok\StringTok{ }\KeywordTok{group_by}\NormalTok{(interval) }\OperatorTok\StringTok{ }\KeywordTok{summarise}\NormalTok{(}\DataTypeTok{steps=}\KeywordTok{mean}\NormalTok{(steps,}\DataTypeTok{na.rm =}\NormalTok{ T))}
\end{Highlighting}
\end{Shaded}

\begin{verbatim}
## `summarise()` ungrouping output (override with `.groups` argument)
\end{verbatim}

\begin{Shaded}
\begin{Highlighting}[]
\KeywordTok{plot}\NormalTok{(activitypatterndata}\OperatorTok{$}\NormalTok{interval,activitypatterndata}\OperatorTok{$}\NormalTok{steps,}\DataTypeTok{type=}\StringTok{"l"}\NormalTok{,}\DataTypeTok{xlab =} \StringTok{"5-minute intervals"}\NormalTok{, }\DataTypeTok{ylab =} \StringTok{"Average number of steps taken"}\NormalTok{)}
\end{Highlighting}
\end{Shaded}

\includegraphics{Activity-monitoring-data_files/figure-latex/unnamed-chunk-4-1.pdf}

The details of interval wtih maximum number of average steps is given
below.

\begin{Shaded}
\begin{Highlighting}[]
\NormalTok{activitypatterndata[}\KeywordTok{which.max}\NormalTok{(activitypatterndata}\OperatorTok{$}\NormalTok{steps),]}
\end{Highlighting}
\end{Shaded}

\begin{verbatim}
## # A tibble: 1 x 2
##   interval steps
##      <int> <dbl>
## 1      835  206.
\end{verbatim}

\subsection{Imputing missing values}\label{imputing-missing-values}

Using the sum function to calculate the total number of missing values
in the ``steps'' variable as ``date'' and ``interval''" have no missing
values.

\begin{Shaded}
\begin{Highlighting}[]
\NormalTok{sumna <-}\StringTok{ }\KeywordTok{sum}\NormalTok{(}\KeywordTok{is.na}\NormalTok{(data}\OperatorTok{$}\NormalTok{steps))}
\end{Highlighting}
\end{Shaded}

The total missing values in the steps variable of the dataset are
\textbf{2304}.

I will now replace the ``NA'' values with the mean value of that
particular interval's remaining values.

\begin{Shaded}
\begin{Highlighting}[]
\NormalTok{replacewithmean <-}\StringTok{ }\ControlFlowTok{function}\NormalTok{(x)\{}\KeywordTok{replace}\NormalTok{(x, }\KeywordTok{is.na}\NormalTok{(x), }\KeywordTok{mean}\NormalTok{(x, }\DataTypeTok{na.rm =} \OtherTok{TRUE}\NormalTok{))\}}
\NormalTok{completedata <-}\StringTok{ }\NormalTok{data }\OperatorTok\StringTok{ }\KeywordTok{group_by}\NormalTok{(interval) }\OperatorTok\StringTok{ }\KeywordTok{mutate}\NormalTok{( }\DataTypeTok{steps =} \KeywordTok{replacewithmean}\NormalTok{(steps))}
\end{Highlighting}
\end{Shaded}

The total number of steps taken each day as per the dataset without
missing values can be seen as below:

\begin{Shaded}
\begin{Highlighting}[]
\NormalTok{plotdatacomplete <-}\StringTok{ }\NormalTok{completedata }\OperatorTok\StringTok{ }\KeywordTok{group_by}\NormalTok{(date) }\OperatorTok\StringTok{ }\KeywordTok{summarise}\NormalTok{(}\DataTypeTok{steps =} \KeywordTok{sum}\NormalTok{(steps))}
\KeywordTok{hist}\NormalTok{(plotdatacomplete}\OperatorTok{$}\NormalTok{steps, }\DataTypeTok{breaks =} \DecValTok{15}\NormalTok{, }\DataTypeTok{xlab =} \StringTok{"Number of steps per day"}\NormalTok{, }\DataTypeTok{main =} \StringTok{"Histogram of daily steps per day"}\NormalTok{, }\DataTypeTok{col =} \StringTok{"blue"}\NormalTok{)}
\KeywordTok{rug}\NormalTok{(plotdatacomplete}\OperatorTok{$}\NormalTok{steps)}
\end{Highlighting}
\end{Shaded}

\includegraphics{Activity-monitoring-data_files/figure-latex/unnamed-chunk-8-1.pdf}

\begin{Shaded}
\begin{Highlighting}[]
\NormalTok{steps_summary_completedata <-}\StringTok{ }\NormalTok{completedata }\OperatorTok\StringTok{ }\KeywordTok{group_by}\NormalTok{(date) }\OperatorTok\StringTok{ }\KeywordTok{summarise}\NormalTok{(}\DataTypeTok{steps =} \KeywordTok{sum}\NormalTok{ (steps))}
\end{Highlighting}
\end{Shaded}

\begin{verbatim}
## `summarise()` ungrouping output (override with `.groups` argument)
\end{verbatim}

\begin{Shaded}
\begin{Highlighting}[]
\NormalTok{meantotal <-}\StringTok{ }\KeywordTok{as.integer}\NormalTok{(}\KeywordTok{mean}\NormalTok{(steps_summary_completedata}\OperatorTok{$}\NormalTok{steps,}\DataTypeTok{na.rm =}\NormalTok{ T))}
\NormalTok{mediantotal <-}\StringTok{ }\KeywordTok{as.integer}\NormalTok{(}\KeywordTok{median}\NormalTok{(steps_summary_completedata}\OperatorTok{$}\NormalTok{steps,}\DataTypeTok{na.rm =}\NormalTok{ T))}
\end{Highlighting}
\end{Shaded}

The mean of steps per day from the dataset without missing values is
\textbf{10766} and the median from the same dataset is \textbf{10766}.

\begin{Shaded}
\begin{Highlighting}[]
\NormalTok{diffmean <-}\StringTok{ }\NormalTok{meantotal }\OperatorTok{-}\StringTok{ }\NormalTok{meansteps}
\NormalTok{diffmedian <-}\StringTok{ }\NormalTok{mediantotal }\OperatorTok{-}\StringTok{ }\NormalTok{mediansteps}
\end{Highlighting}
\end{Shaded}

The difference between means from the two data sets is
\texttt{r}diffmean\texttt{and\ the\ difference\ between\ medians\ from\ the\ two\ data\ sets\ is}r
\texttt{diffmedian}.

\subsection{Difference in activity patterns between weekdays and
weekends}\label{difference-in-activity-patterns-between-weekdays-and-weekends}

\begin{Shaded}
\begin{Highlighting}[]
\NormalTok{completedata}\OperatorTok{$}\NormalTok{day <-}\StringTok{ }\KeywordTok{weekdays}\NormalTok{(completedata}\OperatorTok{$}\NormalTok{date)}
\NormalTok{completedata}\OperatorTok{$}\NormalTok{weekday <-}\StringTok{ }\KeywordTok{ifelse}\NormalTok{(completedata}\OperatorTok{$}\NormalTok{day }\OperatorTok{==}\StringTok{ "Saturday"} \OperatorTok{|}\StringTok{ }\NormalTok{completedata}\OperatorTok{$}\NormalTok{day }\OperatorTok{==}\StringTok{ "Sunday"}\NormalTok{, }\StringTok{"Weekend"}\NormalTok{, }\StringTok{"Weekday"}\NormalTok{)}
\NormalTok{completedata}\OperatorTok{$}\NormalTok{weekday <-}\StringTok{ }\KeywordTok{as.factor}\NormalTok{(completedata}\OperatorTok{$}\NormalTok{weekday)}
\NormalTok{completedata <-}\StringTok{ }\NormalTok{completedata }\OperatorTok\StringTok{ }\KeywordTok{group_by}\NormalTok{(weekday,interval) }\OperatorTok\StringTok{ }\KeywordTok{summarise}\NormalTok{(}\DataTypeTok{steps =} \KeywordTok{mean}\NormalTok{(steps, }\DataTypeTok{na.rm=}\NormalTok{T))}
\end{Highlighting}
\end{Shaded}

\begin{verbatim}
## `summarise()` regrouping output by 'weekday' (override with `.groups` argument)
\end{verbatim}

\begin{Shaded}
\begin{Highlighting}[]
\KeywordTok{qplot}\NormalTok{(}\DataTypeTok{x =}\NormalTok{ interval, }\DataTypeTok{y =}\NormalTok{ steps, }\DataTypeTok{data =}\NormalTok{ completedata, }\DataTypeTok{facets =}\NormalTok{ weekday}\OperatorTok{~}\NormalTok{.,}\DataTypeTok{geom =} \StringTok{"path"}\NormalTok{,}\DataTypeTok{color =}\NormalTok{ weekday)}
\end{Highlighting}
\end{Shaded}

\includegraphics{Activity-monitoring-data_files/figure-latex/unnamed-chunk-11-1.pdf}

\end{document}
